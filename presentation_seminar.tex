\documentclass[xcolor=dvipsnames]{beamer}

  % Core packages
  \usepackage{amsmath, amssymb}
  \usepackage{graphicx}
  \usepackage{booktabs}
  \usepackage{tabularx}
  \usepackage{array}
  \usepackage{multirow}
  \usepackage{threeparttable}
  \usepackage{makecell}
  \usepackage{subcaption}
  \usepackage{natbib}
  \usepackage[table]{xcolor}
  \usepackage{hyperref}

  % Bibliography style (put \bibliography{references} at end of document, not here)
  \bibliographystyle{apalike}

  % Colors
  \definecolor{UBCblue}{rgb}{0.04706, 0.13725, 0.26667}
  \definecolor{UBCgrey}{rgb}{0.3686, 0.5255, 0.6235}

  % Theme
  \usetheme{Madrid}
  \useoutertheme{miniframes}
  \setbeamertemplate{headline}{}
  \useinnertheme{circles}

  % Color scheme
  \setbeamercolor{palette primary}{bg=UBCblue,fg=white}
  \setbeamercolor{palette secondary}{bg=UBCblue,fg=white}
  \setbeamercolor{palette tertiary}{bg=UBCblue,fg=white}
  \setbeamercolor{palette quaternary}{bg=UBCblue,fg=white}
  \setbeamercolor{structure}{fg=UBCblue}
  \setbeamercolor{section in toc}{fg=UBCblue}
  \setbeamercolor{subsection in head/foot}{bg=UBCgrey,fg=white}

  \setbeamertemplate{sidebar right}{}
  \setbeamertemplate{footline}{
    \leavevmode%
    \hbox{%
    \begin{beamercolorbox}[wd=.35\paperwidth,ht=2.25ex,dp=1ex,left]{author in head/foot}%
      \hspace*{2em}\usebeamerfont{author in head/foot}\insertshortauthor~~(\insertshortinstitute)
    \end{beamercolorbox}%
    \begin{beamercolorbox}[wd=.30\paperwidth,ht=2.25ex,dp=1ex,center]{title in head/foot}%
      {\tiny \usebeamerfont{title in head/foot}\insertshorttitle}
    \end{beamercolorbox}%
    \begin{beamercolorbox}[wd=.35\paperwidth,ht=2.25ex,dp=1ex,right]{date in head/foot}%
      \usebeamerfont{date in head/foot}\insertshortdate{}\hspace*{2em}
      \insertframenumber{}\hspace*{2ex}
    \end{beamercolorbox}}%
    \vskip0pt%
  }

\title[Employment Penalty in Japan]{The Dual Labour Market and the Motherhood Employment Penalty in Japan}
\author{Beatriz Gietner}
\institute[UCD]{UCD School of Economics}
\date{\today}

\begin{document}

\maketitle


%% ================================================================
%%  SLIDE 1: THE PUZZLE (1.5 min)
%% ================================================================
 \begin{frame}[allowframebreaks]{The Puzzle}

    Japan offers up to two years of paid parental leave, including one of the longest father-specific entitlements in the OECD (31 weeks) \citep{oecd_pf21_parental_leave}.

    \vspace{1em}
    Yet maternal employment when the youngest child is under three remains far below the OECD benchmark ($\approx$ 30 p.p. in the latest comparison)
  \citep{oecd_lmf12_maternal_employment}.

    \vspace{1em}
    \textbf{If the policy infrastructure exists, why do women still leave?}

    \vspace{1em}
    \normalsize
 In this paper, I show that the key margin is job type: formal eligibility exists, but continuity through childbirth is much weaker in non-regular tracks.
 
 \vspace{1em}
 \footnotesize
 Regular = open-ended, internal career track jobs; non-regular = fixed-term/part-time/dispatch jobs.
  \end{frame}
  

%% ================================================================
%%  SLIDE 2: WHAT THIS PAPER IS AND ISN'T (1 min)
%% ================================================================
\begin{frame}[allowframebreaks]{What This Paper Is and Isn't}
  \textbf{What it is:}
  \begin{itemize}
    \item Descriptive event study of 662 first births (KHPS/JHPS, 2004--2022)
    \item Predictive risk stratification: which jobs predict exit?
    \item Transparent about what is robust, what is fragile, and what is interpretive
  \end{itemize}

  \vspace{1em}
  \textbf{What it is not:}
  \begin{itemize}
    \item Not a causal identification design
    \item Not administrative data (household panel, N=662)
    \item Not nationally weighted (sample-average trajectories)
  \end{itemize}

  \vspace{0.8em}
  \footnotesize
  \textit{If you take one thing away:} the penalty is concentrated in jobs with weaker contractual protections.
\end{frame}


%% ================================================================
%%  SLIDE 3: INSTITUTIONAL CONTEXT (2 min)
%% ================================================================
\begin{frame}[allowframebreaks]{Japan's Dual Labour Market}
  \small
  \begin{columns}[T]
    \begin{column}{0.48\textwidth}
      \textbf{Regular (\textit{seishain})}
      \begin{itemize}
        \item Open-ended contracts
        \item Seniority wages, firm training
        \item Strong de facto job protection
        \item Leave is exercisable
      \end{itemize}
    \end{column}
    \begin{column}{0.48\textwidth}
      \textbf{Non-regular (\textit{hi-seiki})}
      \begin{itemize}
        \item Fixed-term, part-time, dispatch
        \item Flat wages, limited progression
        \item Contracts can lapse at/around leave
        \item Leave is formally available but fragile
      \end{itemize}
    \end{column}
  \end{columns}

  \vspace{1em}
  \begin{itemize}
    \item Over 50\% of employed women hold non-regular contracts
    \item Small firms face tighter staffing, and it's harder to accommodate leave
    \item Spousal tax thresholds (103/130 man-yen) cap re-entry earnings
    \item Husbands work 47+ hrs/week - near-zero paternal caregiving adjustment
  \end{itemize}
  \vspace{0.4em}
  \tiny \textit{Sources:} \citet{oecd2024}, \citet{genda2005}, \citet{piotrowski2015}
\end{frame}


%% ================================================================
%%  SLIDE 4: DATA + DESIGN (2 min)
%% ================================================================
\begin{frame}[allowframebreaks]{Data and Design}
  \small
  \textbf{Data:} Harmonised KHPS/JHPS household panels, 2004-2022. 
  
  \vspace{0.3em}
  \textbf{Sample:} 1,184 first births $\rightarrow$ 662 in event window $\rightarrow$ 5,212 person-years

  \vspace{0.5em}
  \textbf{Specification:}
  \[
  Y_{it} = \sum_{k \neq -1}\beta_k \mathbf{1}\{\text{event\_time}_{it}=k\} + \gamma_t + \varepsilon_{it}
  \]

 \begin{itemize}
\item \(Y_{it}\) is the outcome for woman \(i\) in survey year \(t\); \(\mathbf{1}\{\text{event\_time}_{it}=k\}\) is an indicator equal to 1 when observation \((i,t)\) is at event time \(k\) (and 0 otherwise); \(\beta_k\) are event-time coefficients relative to \(t=-1\); \(\gamma_t\) are calendar-year fixed effects.
\item Event-study around first birth with calendar-year fixed effects (year-FE), heteroskedasticity-robust SEs; woman-clustered SE sensitivity gives the same qualitative inference.
\item No individual FE in the main specification $\rightarrow$ individual FE + year = collinearity and inflated SEs (short unbalanced event-time panel); point
estimates are similar, but inference becomes uninformative.
\item KHPS (since 2004) and JHPS (since 2009) are nationally fielded household panels with harmonized core labour variables; I pool them using the official harmonization framework and verify robustness to survey source (KHPS vs JHPS split).
\item Reference period is \(t=-1\): each coefficient is a difference relative to the survey year before first reported birth
    \item Panel support is unbalanced over. event time (women observed: 186 at \(t=-5\), 662 at \(t=0\), 454 at \(t=+5\)).
  \end{itemize}
  \begin{figure}
    \centering
    \includegraphics[width=0.8\linewidth]{fig_support_by_event_time.png}
      \caption{\footnotesize Event-time support in the unbalanced panel (N women observed by relative year). Support peaks around childbirth and thins in far leads/lags, motivating primary
  interpretation in \([-2,+2]\).}
  \end{figure}

  \vspace{0.5em}
  \textbf{Estimand:} Descriptive sample-average event-time trajectories in this analytic panel (not within-person causal treatment effects).

  \vspace{0.3em}
  \textbf{Interpretation strategy:} Focus on \([-2,+2]\), where support is strongest; treat far leads/lags as composition-sensitive diagnostics.
\end{frame}


%% ================================================================
%%  SLIDE 5: THE BIRTH-YEAR BREAK (3 min)
%% ================================================================
\begin{frame}[allowframebreaks]{Result 1: The Birth-Year Break}
  \begin{figure}
    \centering
    \includegraphics[width=0.78\linewidth]{presentation_employment_levels.png}
  \end{figure}
  \footnotesize
  Employment is flat at $t=-2$. Then it drops from 50.3\% to 38.5\% at birth.\\
  The pre-birth rise is life-cycle entry, not a pre-trend. The far leads are composition, not anticipation.
\end{frame}


%% ================================================================
%%  SLIDE 6: THE RECOVERY IS AN ILLUSION (2 min)
%% ================================================================
\begin{frame}[allowframebreaks]{But the Recovery Is an Illusion}
  \begin{figure}
    \centering
    \includegraphics[width=0.7\linewidth]{fig8_decomposition_shares.png}
  \end{figure}
  \footnotesize
The composition of the penalty flips over time: the extensive margin (employment exit) explains most of the gap right after birth, but by \(t=+3\) the intensive margin (hours/wage losses among returners) becomes dominant and remains so through \(t=+5\).
\end{frame}

%% ================================================================
%%  SLIDE 7: WHO EXITS? (3 min)
%% ================================================================
  \begin{frame}{Result 2a: Who Exits at Childbirth? (Risk Profile)}
  \small
  \textbf{Risk set:} women employed at \(t=-1\), observed at \(t=0\) (N=330; 124 exits).

  \vspace{0.5em}
  \textbf{Short answer: women in non-regular jobs.}

  \begin{itemize}
    \item Exit risk is much lower in protected jobs and much higher in precarious jobs.
    \item Regular workers in large firms: \textbf{13.8\%} exit.
    \item Non-regular workers in small firms: \textbf{64.8\%} exit.
    \item Household-side variables add little once job type is included.
  \end{itemize}
  \end{frame}

  \begin{frame}{Result 2b: Model Summary and Robustness}
  \small
  \textbf{Model summary}
  \begin{itemize}
    \item Non-regular status: strong predictor of childbirth-margin exit (OR \(=7.50\), \(p<0.001\)).
    \item Small-firm effect: positive but less precise (OR \(=1.55\), \(p=0.126\)).
  \end{itemize}

  \vspace{0.5em}
  \textbf{Robustness to missing-data handling}
  \begin{itemize}
    \item Complete-case OR: 7.50
    \item Missing-indicator OR: 7.48
    \item High-information OR: 7.39
  \end{itemize}

  \vspace{0.4em}
  \footnotesize
  \textit{Takeaway:} the non-regular gradient is stable across specifications, so this is not a missing-data artifact.
  \end{frame}

%% ================================================================
%%  SLIDE 8: THE DOWNGRADE --- HOURS AND EARNINGS (2.5 min)
%% ================================================================
  \begin{frame}[allowframebreaks]{Result 3a: Persistent Hours Losses}
    \begin{figure}
      \centering
      \includegraphics[width=0.7\linewidth]{fig2_hours_event_study.png}
    \end{figure}

  \vspace{0.3em}
  \footnotesize
  \textbf{Conditional on employment (relative to \(t=-1\)):}
  \begin{itemize}
    \item Coefficients are changes in weekly hours versus the pre-birth reference year.
    \item By \(t=+5\): \(-10.1\) hours/week, about a 30\% decline from the \(t=-1\) baseline of 34.1.
  \end{itemize}

  \end{frame}
 \begin{frame}[allowframebreaks]{Result 3b: Persistent Earnings Losses}
    \begin{figure}
      \centering
      \includegraphics[width=0.7\linewidth]{fig3_earnings_event_study.png}
    \end{figure}
    \footnotesize
    \textbf{Conditional on employment (relative to \(t=-1\)):}
    \begin{itemize}
      \item Coefficients are changes in annual earnings versus the pre-birth reference year.
      \item Earnings fall more than hours and do not recover within five years.
      \item By \(t=+5\): \(-96.2\) man-yen (about a 36\% decline from the \(t=-1\) baseline of 268.9 man-yen; implied level \(\approx 172.7\)).
      \item Implied hourly wage (earnings divided by annualized hours) also declines (\(-7.1\%\)), consistent with lower-quality re-entry.
    \end{itemize}
    \footnotesize
    Unconditional outcomes (coding 0 when non-employed) show the same qualitative pattern.
  \end{frame}

%% ================================================================
%%  SLIDE 9: EARNINGS BUNCHING (1.5 min)
%% ================================================================
\begin{frame}[allowframebreaks]{Result 4: Earnings Bunching at Tax Thresholds}
  \begin{figure}
    \centering
    \includegraphics[width=1\linewidth]{earnings bunching returners.PNG}
  \end{figure}
 \begin{itemize}
    \item Among employed mothers at \(t=+3\) to \(t=+5\), \textbf{47.8\%} earn \(\leq\) 103 man-yen.
    \item In the same window, \textbf{55.9\%} earn \(\leq\) 130 man-yen.
    \item These thresholds are salient secondary-earner tax/social-insurance benchmarks.
    \item Net-cost jumps around these points can reduce incentives to increase hours, reinforcing low-hours re-entry.
  \end{itemize}

  \footnotesize
  \textit{Notes:} Shares are author calculations from KHPS/JHPS returners at \(t=+3\) to \(t=+5\). Threshold definitions follow \citet{oecd2024} and \citet{nagase2012}.
\end{frame}


%% ================================================================
%%  SLIDE 10: STABILITY MAP (2 min)
%% ================================================================
  \begin{frame}{Evidence Hierarchy: What Is Solid vs Cautious}
  \small
  \textbf{Most credible}
  \begin{itemize}
    \item Sharp childbirth break in employment at \(t=0\) (\(-13.2\)pp).
    \item Near lead at \(t=-2\) is close to zero.
    \item Childbirth-margin exit risk is strongly stratified by non-regular status (OR \(\approx 7.5\)).
  \end{itemize}

  \textbf{Credible but more sample-sensitive}
  \begin{itemize}
    \item Long-horizon (\(t=+5\)) levels and recovery magnitudes.
    \item Small-firm effects conditional on contract type.
  \end{itemize}

  \textbf{Interpretive context (not separately identified)}
  \begin{itemize}
    \item Childcare logistics, paternal-hours narrative, and institutional mechanisms.
  \end{itemize}

  \vspace{0.4em}
  \footnotesize
  Full stability map and all diagnostics are in backup (placebo, IPW, balanced panel, leave recoding, missingness sensitivity, clustered SE, cohort split, Lee-style trimming).
  \end{frame}

%% ================================================================
%%  SLIDE 11: SO WHAT? (1.5 min)
%% ================================================================
\begin{frame}[allowframebreaks]{Implications}
  \textbf{The penalty begins as exit and transforms into permanent downgrade.}

  \vspace{1em}
  Policy implications consistent with the descriptive evidence:
  \begin{enumerate}
    \item \textbf{Protect non-regular workers} - ensure leave rights survive contract renewal
    \item \textbf{Support small firms} - replacement-hiring subsidies, simplified leave admin
    \item \textbf{Address men's overwork} - near-zero paternal adjustment in this sample
    \item \textbf{Break the re-entry trap} - training subsidies are invisible ($<$3\% usage); fiscal thresholds cap earnings
  \end{enumerate}

  \vspace{1em}
  \footnotesize
  \textit{Bottom line:} Expanding childcare and leave helps workers who already hold secure jobs. The binding constraint is labour-market structure - the divide between jobs that can accommodate parenthood and jobs that cannot.
\end{frame}


%% ================================================================
%%  BACKUP SLIDES (not presented unless asked)
%% ================================================================
\begin{frame}[allowframebreaks]{References}
\bibliography{references}
\end{frame}

\begin{frame}[allowframebreaks]{Backup: Summary Statistics at $t=-1$}
  \small
  \begin{tabular}{lrrr}
    \hline\hline
    & Full Sample & Quitters & Stayers \\
    & (N=662) & (N=29) & (N=262) \\
    \hline
    Employed (\%) & 50.3 & 55.2 & 61.0 \\
    Full-time (\%) & 30.9 & 31.0 & 36.7 \\
    Part-time (\%) & 19.4 & 24.1 & 24.3 \\
    Hours/week (mean) & 34.1 & 32.5 & 33.4 \\
    Income, man-yen (mean) & 268.9 & 148.4 & 302.7 \\
    \hline\hline
  \end{tabular}
\end{frame}



\begin{frame}[allowframebreaks]{Backup: Coefficient Event Study}
  \begin{figure}
    \centering
    \includegraphics[width=0.75\linewidth]{fig1_employment_event_study.png}
  \end{figure}
  \footnotesize
  Standard event-study coefficients relative to $t=-1$. Year FE, robust SEs.
\end{frame}

\begin{frame}[allowframebreaks]{Backup: Pre-Trend Diagnostics}
  \small
  The near lead at $t=-2$ is null across four specifications:
  \begin{itemize}
    \item Full sample: $-0.012$, $p=0.70$
    \item Trimmed pre-period (drop $t=-5,-4,-3$): $p=0.72$
    \item Balanced pre-support: $p=0.078$
    \item With KHPS/JHPS dataset indicator: $p=0.71$
  \end{itemize}
  \vspace{0.5em}
  Matched childless placebo: $t=0$ coefficient $= +0.016$, $p=0.608$.\\
  Joint far-lead test: $p=0.749$.
\end{frame}

\begin{frame}[allowframebreaks]{Backup: Cohort Split and COVID}
  \small
  \begin{tabular}{lccc}
    \hline\hline
    & $t=-2$ & $t=0$ & $t=+5$ \\
    \hline
    Pooled & $-0.012$ ($p=0.70$) & $-0.132$ ($p<0.001$) & $-0.028$ ($p=0.38$) \\
    Early ($\leq$2012) & $+0.020$ ($p=0.63$) & $-0.131$ ($p<0.001$) & $-0.020$ ($p=0.66$) \\
    Late ($\geq$2013) & $-0.063$ ($p=0.19$) & $-0.135$ ($p=0.002$) & $-0.025$ ($p=0.66$) \\
    No-COVID & $-0.004$ ($p=0.91$) & $-0.145$ ($p<0.001$) & $-0.028$ ($p=0.40$) \\
    \hline\hline
  \end{tabular}
  \vspace{0.5em}

  The birth-year break is virtually identical across cohorts and unaffected by excluding 2020--2022.
\end{frame}

\begin{frame}[allowframebreaks]{Backup: IPW and Attrition}
  \small
  \begin{itemize}
    \item Observability at $t=+5$ is weakly selective on baseline employment ($p=0.045$)
    \item IPW reweighting:
    \begin{itemize}
      \item Near lead: unchanged
      \item Birth-year break: unchanged ($-0.136$, $p<0.001$)
      \item $t=+5$: shifts to $-0.057$ ($p=0.076$) - more negative
    \end{itemize}
    \item Balanced panel ($t=-2$ to $t=+1$, N=662, 2,648 obs): core coefficients preserved
  \end{itemize}
\end{frame}

\begin{frame}[allowframebreaks]{Backup: Unconditional Outcomes}
  \small
  \input{event_study_unconditional_key_coeffs.tex}
  \vspace{0.3em}

  \footnotesize
  Coding 0 for non-employed preserves the post-birth declines and near-lead null.
\end{frame}

\begin{frame}[allowframebreaks]{Backup: Missingness Sensitivity (Mechanism)}
  \small
  \input{t0_missingness_profile.tex}
  \vspace{0.3em}
  \input{t0_mechanism_missingness_sensitivity.tex}
  \vspace{0.3em}

  \footnotesize
  Non-regular OR stable: 7.50, 7.48, 7.39 across specifications.
\end{frame}

\begin{frame}[allowframebreaks]{Backup: Pre-Birth Mechanism ($t=-2$, 13 events)}
  \small
  \begin{tabular}{lcccc}
    \hline\hline
    & OR & $p$ & 95\% CI & Method \\
    \hline
    \multicolumn{5}{l}{\emph{Non-regular}} \\
    Standard & 6.41 & 0.019 & - & MLE \\
    Firth & 5.32 & 0.009 & - & Penalised \\
    Bootstrap & - & 0.008 & [1.68, 19.66] & 2,000 \\
    \hline
    \multicolumn{5}{l}{\emph{Small firm}} \\
    Standard & 4.14 & 0.025 & - & MLE \\
    Firth & 3.84 & 0.021 & - & Penalised \\
    Bootstrap & - & 0.038 & [1.10, 18.34] & 2,000 \\
    \hline\hline
  \end{tabular}
  \vspace{0.3em}

  \footnotesize
  Jackknife: non-regular OR above 1 in all 185 replications (range: 5.69--12.82).
\end{frame}

\begin{frame}[allowframebreaks]{Backup: Father-Side Evidence}
  \begin{figure}
    \centering
    \includegraphics[width=0.8\linewidth]{fig5_father_penalty.png}
  \end{figure}
  \footnotesize
  Flat profile. Childcare ratio 6.3:1 at $t=+1$, 5.3:1 at $t=+5$.\\
  Only 3/1,183 husbands (0.3\%) took childcare leave.
\end{frame}

\begin{frame}[allowframebreaks]{Backup: Training Grants Usage}
  \small
  \begin{tabular}{rrrr}
    \hline\hline
    Event time & N & Skills training (\%) & Grants used (\%) \\
    \hline
    $-2$ & 253 & 35.2 & 2.2 \\
    $-1$ & 475 & 31.2 & 2.3 \\
    $0$  & 568 & 8.6  & 1.0 \\
    $+1$ & 594 & 10.6 & 1.3 \\
    $+3$ & 493 & 15.6 & 0.3 \\
    $+5$ & 437 & 12.8 & 0.0 \\
    \hline\hline
  \end{tabular}
  \vspace{0.3em}

  \footnotesize
  Training drops from $\sim$35\% to 8.6\% at birth, recovers to $\sim$13\%.\\
  Government grants: $<$3\% at every horizon. System is invisible.
\end{frame}

\begin{frame}[allowframebreaks]{Backup: KHPS/JHPS Split}
  \small
  \input{survey_split_stability_eventstudy.tex}
  \vspace{0.3em}

  \footnotesize
  Near lead and birth-year break are stable across surveys. Long horizon diverges.
\end{frame}

\begin{frame}[allowframebreaks]{Backup: Heterogeneity by Contract Type}
  \begin{figure}
    \centering
    \includegraphics[width=0.9\linewidth]{fig9a_heterogeneity_jobtype.png}
  \end{figure}
  \footnotesize
  Non-regular: $-9.8$ pp at $t=0$. Regular: $-1.6$ pp. Five-fold difference.
\end{frame}

\begin{frame}[allowframebreaks]{Backup: Earnings Bunching by Event Time}
  \begin{figure}
    \centering
    \includegraphics[width=1\linewidth]{earnings bunching returners by event time.PNG}
  \end{figure}
\end{frame}

\begin{frame}[allowframebreaks]{Backup: Geographic Non-Variation}
  \small
  \begin{itemize}
    \item Metropolitan (Kanto/Kinki, N=355): $-13.6$ pp at $t=0$
    \item Non-metropolitan (N=307): $-12.9$ pp at $t=0$
    \item Near lead null in both subsamples
  \end{itemize}
  \vspace{0.5em}
  The dual labour market operates nationwide.
\end{frame}

\begin{frame}[allowframebreaks]{Backup: Leave-Recoding Sensitivity}
  \small
  \input{leave_sensitivity_t0_table.tex}
  \vspace{0.3em}

  \footnotesize
  Reclassifying 22 leave-takers as employed attenuates $t=0$ from $-0.132$ to $-0.099$; remains strongly significant.
\end{frame}

\begin{frame}[allowframebreaks]{Backup: What Is Robust, What Isn't}
  \footnotesize
  \begin{tabular}{p{0.30\linewidth} p{0.30\linewidth} p{0.30\linewidth}}
    \hline\hline
    \textbf{Stable} & \textbf{Sensitive} & \textbf{Interpretive} \\
    \hline
    Near lead null & Long horizon & Marriage-stage \\[-0.2em]
    {\tiny ($-0.012$, $p=0.70$)} & {\tiny (composition-sensitive)} & {\tiny sequencing} \\[0.3em]
    Birth-year drop & Far leads reflect & Childcare-logistics \\[-0.2em]
    {\tiny ($-0.132$, $p<0.001$)} & {\tiny changing support} & {\tiny narrative} \\[0.3em]
    Non-regular OR at $t\!=\!0$ & Pre-birth mechanism & Household-structure \\[-0.2em]
    {\tiny (7.50, $p<0.001$)} & {\tiny (13 events)} & {\tiny channels} \\[0.3em]
    Cohort-split stable & Industry splits & Cross-country \\[-0.2em]
    {\tiny (early $\approx$ late $\approx$ no-COVID)} & {\tiny selection-sensitive} & {\tiny framing} \\[0.3em]
    Bunching at 103 man-yen & Hours/earnings cond. & Training non-uptake \\[-0.2em]
    {\tiny (47.8\%)} & {\tiny on employment} & {\tiny causes} \\
    \hline\hline
  \end{tabular}

  \vspace{0.5em}
  \normalsize
  Additional checks in backup: placebo, IPW, balanced panel, Lee bounds, clustered SEs, KHPS/JHPS split, missingness sensitivity, jackknife.
\end{frame}


\end{document}
